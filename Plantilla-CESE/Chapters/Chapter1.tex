% Chapter 1

\chapter{Introducción General} % Main chapter title
En este capítulo se menciona la problemática que motivó la realización del presente trabajo, los procedimientos de la toma de datos en la industria actualmente y el enfoque elegido para desarrollar el prototipo que ofrece una solución usando sistemas embebidos.
\label{Chapter1} % For referencing the chapter elsewhere, use \ref{Chapter1} 
\label{IntroGeneral}

%----------------------------------------------------------------------------------------

% Define some commands to keep the formatting separated from the content 
\newcommand{\keyword}[1]{\textbf{#1}}
\newcommand{\tabhead}[1]{\textbf{#1}}
\newcommand{\code}[1]{\texttt{#1}}
\newcommand{\file}[1]{\texttt{\bfseries#1}}
\newcommand{\option}[1]{\texttt{\itshape#1}}
\newcommand{\grados}{$^{\circ}$}

%----------------------------------------------------------------------------------------

%\section{Introducción}

%----------------------------------------------------------------------------------------

%http://jderobot.org/store/juancamilo/uploads/tesismaster.pdf
\section{Motivación}
En el transcurso de la próxima década se espera un gran crecimiento en la cantidad de dispositivos IoT (\textit{internet of things}) provenientes de redes LPWAN (\textit{Low-Power Wide Area Network}). Para el 2025, se espera que más de 100 billones de dispositivos se conecten a través de LPWAN\cite{taylor2015world}. Las principales tecnologías, que prometen una vida útil alta de la batería de los dispositivos y un alcance de hasta 15 kilómetros, son Sigfox, Lora y NB-IoT, que actualmente están conectados en todo el mundo con más de 25 millones de dispositivos, brindando servicio y facilitando las experiencias del usuario\citep{iotanalytics}.

En los procesos industriales se tiene mucha información de variables físicas y eléctricas por medio de sensores, la cual muchas veces no se aprovecha debido a que no se tiene una óptima trazabilidad de la misma o simplemente se pierde esta información. Cuando los procesos industriales fallan, necesitan una reacción inmediata por parte de una persona y esto genera una dependencia de alguien que no siempre va a estar las 24 horas del día al pendiente, quizá por costos para las mismas empresas.

En vista de lo de anterior se desarrolló un prototipo para ofrecer una solución que permita monitorear inalambricamente diferentes variables en los procesos industriales de esta manera los usuarios pueden estar informados y garantizan la correcta funcionalidad de sus procesos.

%https://iot-analytics.com/state-of-the-iot-update-q1-q2-2018-number-of-iot-devices-now-7b/


\section{IoT (\textit{Internet of Things})}
El concepto de internet de las cosas se refiere a la interconexión digital de dispositivos y objetos  a través  de una red, es decir, dispositivos como sensores y/o actuadores, equipados con una interfaz de comunicación, unidades de procesamiento y almacenamiento\cite{centenaro2016long}. Estos dispositivos tienen la capacidad de adquirir, intercambiar y transferir datos a la red mediante alguna tecnología de comunicación inalámbrica.



IoT es una tendencia imparable y puede facilitar mucho la vida diaria. Produce formas baratas y efectivas de resolver grandes problemas sociales, como el acceso a la energía, el transporte y la vivienda. Otras aplicaciones pueden ser \textit{wearables}, construcciones y demóticas, \textit{smart cities}, \textit{smart manufacturing}\citep{taylor2015world}. IoT puede hacernos sentir más cómodos en nuestros hogares y en nuestras ciudades.

\subsection{Tecnologías de comunicación}
Uno de los principales habilitadores de un proyecto de internet de las cosas son las redes de comunicaciones. Estas permiten conectar dispositivos, máquinas, sensores o “cosas” los cuales generan datos o información desde cualquier punto geográfico del planeta. Las redes de comunicación son un conjunto de medios técnicos que permiten la comunicación entre equipos que se encuentran a distancia.

Las principales características de una red de comunicación IoT son:
\begin{itemize}
	\item Baja tasa de datos.
	\item Bajo consumo de energía.
	\item Largo alcance de comunicación.
	\item Conexiones bidireccionales.
	\item Movilidad y servicios de localización.

\end{itemize}

En la tabla \ref{tab:Tecno} se puede observar una comparación de las principales tecnologías de comunicación.

\begin{table}[h]
%\small
	\centering
	\caption[Redes de comunicación]{Redes de comunicación más utilizadas para proyectos IoT}
	\begin{tabular}{l c c c}    
		\toprule
		\textbf{Tecnología} 	 & \textbf{Consumo}  & \textbf{Alcance} 	& \textbf{Tasa de Datos} \\
		\midrule
		GSM/GPRS				 & Muy alto			& Alto					&	Alta \\		
		SigFox					 & Bajo				& Medio/alto			&	Muy baja \\
		Lora					 & Bajo				& Medio/alto			&	Muy baja\\	
		WiFi					 & Alto				& Bajo					&	Muy alta \\
		BLE					 	 & Muy bajo			& Muy Bajo				&	Baja \\
		ZigBee					 & Medio			& Bajo					&	Baja \\	
		\bottomrule
		\hline
	\end{tabular}
	\label{tab:Tecno}
\end{table}

\textbf{Tecnología GSM/GPRS:}

GSM (\textit{Global System for Mobile communications)} o en español sistema global para las comunicaciones móviles y es un tipo de red que se utiliza para la transmisión móvil de voz y datos.

GPRS (\textit{General Packet Radio Service)} o en español servicio general de paquetes vía radio y es una extensión mejorada del GSM.
Permite la mensajería instantánea, los servicios de mensajes cortos SMS (\textit{Short Message Service}), multimedia MMS (\textit{Multimedia Messaging Service}) y correo electrónico. Esta proporciona una cobertura inalámbrica completa, tiempos de acceso mas cortos y mayores tasas de datos\citep{bettstetter1999gsm}. Por ejemplo, permite enviar 30 SMS por minuto, mientras que con GSM se puede enviar entre 6 y 10.

\textbf{Tecnología WiFi:}

Es una tecnología que permite la interconexión inalámbrica de dispositivos electrónicos por medio de internet. WiFi, el nombre popular para el área local inalámbrica Redes basadas en el estándar IEEE 802.11b, se ha convertido en la Tecnología preferida para redes inalámbricas de área local en entornos comerciales y domésticos\citep{henry2002wifi}.

\textbf{Tecnología BLE:}

Es una tecnología de red de área personal PAN (\textit{Personal Area Network}) inalámbrica, Permite la comunicación entre dispositivos dos o  más dispositivos Bluetooth, que opera en 2.4 GHz (una de las bandas ISM), con una tasa de transferencia de 1 Mbps en la capa física. BLE (\textit{Bluetooth Low Energy}) se introdujo por primera vez en 2010 con el objetivo de expandir la aplicación de Bluetooth para su uso en dispositivos con limitaciones de energía, como los inalámbricos. Sensores y controles inalámbricos. Los sensores y controles requieren un bajo consumo de energía, pero la cantidad de transmisión de datos es pequeña y la comunicación ocurre con poca frecuencia\citep{chang2014bluetooth}.


\textbf{Tecnología ZigBee:}

ZigBee es uno de El transceptor estándar más utilizado en sensores inalámbricos.redes ZigBee sobre IEEE 802.15.4 , define especificaciones para baja velocidad de datos WPAN (\textit{wireless personal area network}) para soportar baja potencia en monitorización y control de dispositivos\citep{ramya2011study}.

El consumo de energía para ZigBee es muy pequeño. En la mayoría de los casos Utiliza 1mW (o menos potencia). Pero aún así proporciona un alcance hasta 150 metros en exterior que se consigue con la técnica.llamado espectro de propagación de secuencia directa DSSS (\textit{direct sequence spread spectrum}).Funciona en los 868 MHz (Europa), 915 MHz (América del Norte y Australia) y 2.4 GHz (disponible en todo el mundo) banda ISM con hasta 20kbps, 40kbps y velocidad de datos de 250kbps respectivamente\citep{ramya2011study}.

\textbf{Tecnología NB-IoT (\textit{Narrowband Internet of Things}:}

Tecnología de acceso por radio que proporciona cobertura extendida, alta capacidad y larga duración de la batería. Utiliza la ya existente red móvil para conectar dispositivos de manera masiva.

NB-IoT requiere un ancho de banda mínimo de 180 kHz, que es igual al tamaño del LTE físico más pequeño.
Dependiendo de la disponibilidad del espectro, esta tecnología se puede implementar por sí solo en los portadores de guardia de LTE / UMTS existentes\citep{adhikary2016performance}.

Permite transmisiones de hasta 1600 bytes sin limitación de cantidad mensajes por día, con un alcance de 1km en zonas urbanas y 10 km en zonas rurales\cite{MEKKI20191}.

\textbf{Tecnología SigFox:} 

Es una tecnología de comunicación UNB (\textit{Ultra-Narrow Band}) para conectar sensores y dispositivos. Opera en las bandas 868 MHz y 902-928 MHz, permite la transmisión de datos hasta 12 bytes con limitación de 140 mensajes de subida y 4 mensajes de bajada, esta tecnología tiene un alcance de hasta 10 km en zonas urbanas y 40 km en zonas rurales\cite{MEKKI20191}.

\textbf{Tecnología LoRa:} 

Es una tecnología LPWAN de modulación de radio de CCS (\textit{chirp spread spectrum}). Esta permite el envió y recepción de información en las bandas de frecuencia 433 MHz, 868 MHZ y 915 MHz, permite la transmisión de hasta 243 bytes sin limitación de cantidad de mensajes por día y tiene un alcance de 5km en zonas urbanas y 20 km en zonas rurales\cite{MEKKI20191}.

En la tabla \ref{tab:Comparacion1Tabla} se puede observar las comparaciones de costos de las principales tecnologías LPWAN para el IoT : Sigfox, NB-IoT y LoRaWAN. En ella se muestra una ventaja en relación a costos comparado con NB-IoT\cite{mekki2018overview}.

\begin{table}[h]
    \small
	\centering
	\caption[Comparación costos.]{Comparación costos de tecnologías LPWAN para el IoT.}
	\begin{tabular}{l c c c}    
		\toprule
		\textbf{ } 	   & \textbf{Spectrum cost} & \textbf{Deployment cost} 	& \textbf{End-device cost}\\
		\midrule
		Sigfox 	    &Free &>4000 euro/base station &<2 euros\\	
 		LoRaWAN 	&Free  &>100 euros/gateway y >1000 euros/base station &3-5 euros\\
        NB-IoT      &>500 M euro /MHz &>15000 euro/base station &>20 euros\\
		\bottomrule
		\hline
	\end{tabular}
	\label{tab:Comparacion1Tabla}
\end{table}

Las tecnologías seleccionadas fueron Sigfox y LoRaWAN, para la selección de estas tecnologías se tuvo en cuenta algunas características tales como : LPWAN con largo alcance\cite{Samie:2016:ITE:2968456.2974004} y bajo costo\cite{mekki2018overview}. En el capitulo \ref{Chapter2} se explicaran en detalle estas tecnologías .


%---------------Objetivos y alcance 
\section{Objetivos y alcance}

\subsection{Objetivo}
El objetivo principal es diseñar e implementar un dispositivo de adquisición de datos con múltiples entradas digitales y analógicas para aplicaciones IoT en ambientes industriales, mediante la transmisión inalámbrica de la información por medio de tecnologías de comunicación Sigfox o Lora. 

\subsection{Alcance}

En la presente solución se contempla:


\begin{itemize}
	\item La implementación de un prototipo funcional de hardware.
	\item 2 entradas analógicas de tensión.
	\item 1 entrada analógica de corriente.
	\item 5 entradas digitales.
	\item La escritura del firmware del dispositivo.
	\item La transmisión de la información por medio de Sigfox.
	\item La transmisión de la información por medio de Lora.
	\item La visualización de la información en una plataforma paga o libre.
	\item Se incluye partes del código de la biblioteca usada para el modulo Sigfox.
\end{itemize}
En la presente solución no se contempla:
\begin{itemize}
	\item El desarrollo de la plataforma web que permite visualizar los datos en linea.
	\item Caja plástica del dispositivo.
\end{itemize}
En la presente solución no se incluye:
\begin{itemize}
	\item Diagramas esquemáticos.
	\item PCB \textit{layout}.
	\item Firmware.
\end{itemize}

Esto debido a que la propiedad intelectual es de Tecrea SAS.


%----------------------------------------------------------------------------------------

