% Chapter 1

\chapter{Introducción General} % Main chapter title
En este capítulo se menciona la problematica que motivo la realización del presente trabajo, los procedimientos de la toma de datos en la industría actualmente y el enfoque elegido para desarrollar el prototipo que ofrece una solución usando sistemas embebidos.
\label{Chapter1} % For referencing the chapter elsewhere, use \ref{Chapter1} 
\label{IntroGeneral}

%----------------------------------------------------------------------------------------

% Define some commands to keep the formatting separated from the content 
\newcommand{\keyword}[1]{\textbf{#1}}
\newcommand{\tabhead}[1]{\textbf{#1}}
\newcommand{\code}[1]{\texttt{#1}}
\newcommand{\file}[1]{\texttt{\bfseries#1}}
\newcommand{\option}[1]{\texttt{\itshape#1}}
\newcommand{\grados}{$^{\circ}$}

%----------------------------------------------------------------------------------------

%\section{Introducción}

%----------------------------------------------------------------------------------------
\section{Motivación}
En el transcurso de la próxima decada se espera un gran crecimiento en la cantidad de dispositivos IoT provenientes de redes LPWAN (\textit{Low-Power Wide Area Network}). Para el 2025, se espera que más de 2 mil millones de dispositivos se conecten a través de LPWAN. Las principales tecnologías, que prometen una vida útil alta de la batería de los dispositivos y un alcance de hasta 15 kilómetros, son Sigfox, Lora y NB-IoT, que actualmente están conectados en todo el mundo con más de 25 millones de dispositivos, brindando servicio y facilitando las experiencias del usuario.


En los procesos industriales se tienen mucha información de variables físicas y eléctricas por medio de sensores la cual muchas veces no se aprovecha debido a que no se tiene una optima trazabilidad de la misma o simplemente se pierde esta información, también pasa que en los procesos que son críticos cuando fallan necesitan una reacción inmediata por parte de una persona y esto nos genera una dependencia de alguien que no siempre va a estar las 24 horas del día al pendiente, quizá por costos para las mismas empresas. En vista de lo de anterior se desarrolló un prototipo para ofrecer una solución que permita monitorear inalambricamente diferentes variables en los procesos industriales, La cual trasmite a una plataforma donde esta hace lo mismo a los usuarios interesados para que puedan estar informados y garanticen la correcta funcionalidad de sus procesos.

%https://iot-analytics.com/state-of-the-iot-update-q1-q2-2018-number-of-iot-devices-now-7b/
\section{Objetivos y alcance}

\subsection{Objetivo}
El objetivo principal es diseñar e implementar un dispositivo de adquisición de datos con múltiples entradas digitales y analógicas para aplicaciones IoT en ambientes industriales, mediante la transmisión inalámbrica de la información por medio de tecnologías de comunicación Sigfox o Lora. 

\subsection{Alcance}

En la presente solución se contempla:


\begin{itemize}
	\item La implementación de un prototipo funcional de hardware.
	\item 2 entradas analógicas de tensión.
	\item 1 entrada analógica de corriente.
	\item 5 entradas digitales.
	\item La escritura del firmware del dispositivo.
	\item La transmisión de la información por medio de Sigfox.
	\item La transmisión de la información por medio de Lora.\emph{Esto depende del gateway TBD}
	\item La visualización de la información en una plataforma paga o libre.
	\item Se incluye partes del código de la biblioteca usada para el modulo Sigfox.
\end{itemize}
En la presente solución no se contempla:
\begin{itemize}
	\item El desarrollo de la plataforma web que permite visualizar los datos en linea.
	\item Caja plástica del dispositivo.
\end{itemize}
En La presente solución no se incluye:
\begin{itemize}
	\item Diagramas esquemáticos.
	\item PCB \textit{layout}.
	\item Firmware.
\end{itemize}

Esto debido a que hacen parte de la propiedad intelectual de Tecrea SAS.

%----------------------------------------------------------------------------------------






